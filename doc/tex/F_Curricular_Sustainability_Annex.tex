\apendice{Curricular Sustainability Annex}

\section{Introduction}
Sustainable development is an aspect that can determine both a projects longevity, as well as its importance on society. Its application can make or break a project and its importance has been recognized in this work. It can be best summed up as a projects ability to meet the needs of present requirements, without compromising its ability to be used, maintained or expanded in the future \cite{iisd:susDev}.

A few techniques that have been employed in this project that can be seen as sustainable are the following:
\begin{itemize}
    \item \textbf{Open source development}: All the code used for the project can be found inside the public GitHub repository \cite{github:repo}. By publicly sharing the source code to everyone, we can help and encourage future projects by reducing their cost spent in buying either private sources or investing in private works. It also allows for free code that can be used as an example to study and learn from. Whether that be spotting mistakes, extending the code or adapting the code to different problems.
    \item \textbf{Code Comments}: By documenting simple and efficient code comments inside the project, future developers can more effectively understand the code's function and how to use it. It allows for easier learning and and maintenance, as well as paving the way for future extensions.
    \item \textbf{Split code structure}: Inside the scripts, the individual functions that are used are clearly separated and commented, their function calls summarize what the function does, therefore increasing readability as well as making the process easier to understand. This will help future developers understand the code faster and how the functions are used.
    \item \textbf{Popular frameworks}: By using more commonly used frameworks, we make our application more manageable as the documentation and usage of these frameworks is more frequent, therefore bringing a higher number of users that have interacted and can interact with this project. Additionally, these frameworks allow for faster response times to questions asked on forums as there will be more active users. All of this makes the project more maintainable and allows for easier extension.
    \item \textbf{Framework Accessibility and ease of use}: It also helps sustainability, that MATLAB is a very user friendly in terms of its accessibility, especially when it comes to the area of this project, which is the usage of artificial neural networks. Especially for ANNs, MATLAB offers many examples and code explanations as well as video tutorials that all help make the code easier to understand. MATLABs accessibility also makes it a great option to perform experiments on, as going through the code step by step can severely help with understanding code sequences. By not having to go out of your way to go into debugging mode, the visibility and readability of the steps are a lot easier.
\end{itemize}
Unfortunately there is a part of this project that hurts the sustainability, and that is the requirement of a MATLAB license. As it is needed to run and edit the scripts properly, the purchase of the license can't be worked around without completely changing the working environment. Fortunately, as mention in appendix E in the software requirements, the acquisition of the MATLAB license isn't as difficult as it could be. With the addition of student discounts and lower prices for home users\footnote{The license cost comes to around 69€ for students and 119€ for home usage}, buying a MATLAB license (which is a one time cost, for a permanent software) can be worth price.

The work itself, can also help contribute to society in a variety of ways that help overall sustainability and the environment. By being able to classify sky topology under the CIE standard, the usage of natural light and solar energy can be optimized for every location, helping reduce energy usage globally.
\imagen{sustain}{Example graphic of components that can make up sustainable code}{}