\capitulo{1}{Introduction}
The importance of sky classification can be seen in various fields such as architectural design\cite{LI2014563}\cite{ALSHAIBANI2017387}, environmental studies\cite{LI2001435} and the installation of solar energy\cite{kasht_2018}. The traditional way of achieving the sky classification is under the CIE (Commission internationale de l’Eclairage) standard: CIE S 011/E:2003\cite{CIE-Std}.The CIE standard defines a set of 15 sky types. For each one, an angular distribution of luminance\footnote{Luminance being the intensity of visible light emission} is proposed.

One traditional method of performing this sky classification, is through the use of a sky scanner, which is a device that measures the skies luminance and applies mathematical formulas to the acquired measurements to determine the sky's characteristics\cite{tregenza}.
The results of these measurements and calculations are very accurate, however, they are very expensive \cite{GARCIA2022147}. Another downside, aside from the high cost, is that the sky scanner has a very low spatial resolution of 145 measurements and a temporal resolution of 5 minutes per scan\cite{skyClassANN-Granados-Lopéz}.

In order to alleviate some of these issues, this project proposes the usage of an Artificial Neural Network (ANN), which, in recent years, have emerged as a popular alternative to help with the ever increasing abundance of data. It's use for data mining comes from its fast and autonomous computation of large sets of data, while delivering good results. These ANNs are computational models, inspired by the complexity of the human brain, with the capacity of learning from data and making predictions or classifications based patterns detected within the learned data. 

As aforementioned, ANNs can handle large datasets rapidly and efficiently, with the capability of delivering fast predictions/classifications once the network has been trained. Furthermore, ANNs offer a dynamic and adaptive way to handle data from various diverse environments and conditions, as they can simply be trained for its required environment. As we also don't require the purchase of a sky scanner, the saved costs of the initial purchase and further future cost of maintenance can be completely saved by using an ANN. It is also important to mention that the acquisition of a high quality sky scanner for sky luminance data gathering is scarce\cite{skyClassANN-Granados-Lopéz}. 

There are however a few initial downsides to this alternative. In order for the ANN to classify accurately, it requires an extensive dataset of sky images for each sky type defined in the CIE standard. To do this, a digital camera must be set up to capture sky images through a fish-eye lense\cite{skyClassANN-Granados-Lopéz}. The camera will take pictures at a regular interval, while a sky scanner performs scans of the same sky condition and classifies it. The classification process of the scanner is automatic and fast. Once enough images and scans have been acquired, the sky type labelling by the sky scanner needs to be evaluated. This is very time consuming, as the predictions are either compared through visual inspection of the data and/or by comparing the luminance prediction. Luminance prediction can be done using a statistical index, such as the root mean square error. Visual inspection can be very tedious, as it must be done one sample at a time. Data acquisition is the most time consuming part of using an ANN, as the dataset collection and preprocessing is one of the most influential factors in how well the ANN will classify in the end, therefore it is most important to focus on a well maintained and consistent database.

Nevertheless, once a functional ANN has been obtained, its potential use cases can be endless. ANNs could be integrated inside cities for urban monitoring systems, which dynamically adjust the street lighting in accordance to the current sky conditions, thus improving energy efficiency and safety. \cite{ANN-Applications} ANNs could also help with solar energy farms, as it could increase the accuracy of solar power generation forecasts, as it provides precise data on the current sky condition. For environmental studies, the ANN can help forward the studies of sky conditions on local ecosystems and climate. Architects may also use ANNs to make a detailed analysis of the sky conditions for specific locations, in order to assist with the design of buildings, to optimize their usage of natural lighting, which leads to less energy usage by inhabitants.

Previous research done on this topic has focused on image pre-processing to improve ANN performance using the Scaled Conjugate Gradient method for network training. \cite{skyClassANN-Granados-Lopéz} This project however, will be focusing on alternative training methods, that may achieve similar or even better results than the aforementioned training method. We will thus be looking at different training methods and their classification results. The chosen training methods are Resilient Backpropagation (RP), Gradient Descent with Momentum and Adaptive Learning Rate Backpropagation (GDX), and One-step Secant (OSS). 

RP is the fastest algorithm on pattern recognition problems and requires lower memory requirements when compared to other algorithms for these types of problems. GDX is usually slower than the other methods, but can still be useful for some problems. Its slower convergence can come in handy for some situations. OSS serves as a bridge alternative between 2 different algorithms and brings its own advantages and disadvantages, which will be shortly touched upon in the "Techniques and Tools" sections. SCG, which was used in the previous work, is a very balanced algorithm that can perform well over a wide variety of problems and is relatively fast on pattern recognition problems \cite{matlab:tfunc}. It will be used in this work as a comparative measure that has been tried, tested and proven to be of use in sky classifications using ANNs.